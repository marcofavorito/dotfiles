% !TeX root = thesis.tex
% !TeX spellcheck = en_US
% !TeX encoding = UTF-8
\documentclass[12pt]{article}

\title{A title. \LaTeXe{}}
\author{
        Marco Favorito \\
                Department of Computer, Control, and Management Engineering\\
                Sapienza University of Rome\\
                Via Ariosto 25\\ 
                00185 Roma, Italy 
}
\date{\today}


\usepackage{biblatex}
\usepackage[utf8]{inputenc}
\usepackage{csquotes}
\usepackage[english]{babel}
\usepackage{xspace}
\usepackage{amsmath}
\usepackage{amssymb}
\usepackage{graphicx}

\addbibresource{main.bib}

\begin{document}
\input{macros}
\maketitle

\begin{abstract}
This is the paper's abstract \ldots
\end{abstract}

\section{Introduction}
This is time for all good men to come to the aid of their party!

Ths document is an example of BibTeX using in bibliography management. Three items 
are cited: \textit{The \LaTeX\ Companion} book \cite{latexcompanion}, the Einstein
journal paper \cite{einstein}, and the Donald Knuth's website \cite{knuthwebsite}. 
The \LaTeX\ related items are \cite{latexcompanion,knuthwebsite}. 
 
\paragraph{Outline}
The remainder of this article is organized as follows.
Section~\ref{previous work} gives account of previous work.
Our new and exciting results are described in Section~\ref{results}.
Finally, Section~\ref{conclusions} gives the conclusions.

\section{Previous work}\label{previous work}
A much longer \LaTeXe{} example was written by Einstein\cite{einstein}.

\section{Results}\label{results}
In this section we describe the results.

\section{Conclusions}\label{conclusions}
We worked hard, and achieved very little.

\printbibliography

\end{document}
